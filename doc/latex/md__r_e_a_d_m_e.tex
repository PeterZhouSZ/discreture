This is a modern C++ 11 (and 14) library designed to facilitate combinatorial research by providing fast and easy iterators to a few combinatorial objects, such as combinations, permutations, partitions, and others. The idea is to have them resemble the S\-T\-L containers as much as possible, without actually storing the whole set of objects in memory.

Discreture is designed to follow the S\-T\-L containers as closely as possible, by providing the standard ways of iterating. In addition, many of the algorithm described in the standard $<$algorithm$>$ work as-\/is in these containers, but they should be treated as const containers.

\section*{Example use\-:}

```c++ \#include $<$iostream$>$ \#include \char`\"{}discreture.\-hpp\char`\"{} using namespace std; using namespace dscr; int main() \{ combinations X(5,3); for (const auto\& x \-: X) cout $<$$<$ x $<$$<$ endl; return 0; \} ``` The above code would produce the following output\-: \begin{DoxyVerb}[ 0 1 2 ]
[ 0 1 3 ]
[ 0 2 3 ]
[ 1 2 3 ]
[ 0 1 4 ]
[ 0 2 4 ]
[ 1 2 4 ]
[ 0 3 4 ]
[ 1 3 4 ]
[ 2 3 4 ]
\end{DoxyVerb}


Of course, you need to link with the discreture library\-: g++ -\/\-O3 -\/ldiscreture main.\-cpp

Some tests show discreture is usually faster when compiled with clang++ instead of g++.

\section*{Installation}

To download and install, run the following commands\-:

```sh git clone \href{https://github.com/mraggi/discreture.git}{\tt https\-://github.\-com/mraggi/discreture.\-git} cd discreture mkdir build cd build cmake .. make sudo make install \#optional ```

You can run tests like this\-: {\ttfamily ./testdiscreture}

\section*{Combinatorial Objects}

There are a few combinatorial objects, such as\-:
\begin{DoxyItemize}
\item Combinations
\item Permutations
\item Subsets
\item Multisets
\item Partitions
\item Dyck Paths
\item Range
\item Motzkin Paths
\end{DoxyItemize}

These all follow the same design principle\-: The templated class is calles basic\-\_\-\-S\-O\-M\-E\-T\-H\-I\-N\-G$<$class T$>$, and the most reasonable type for T is instantiated as S\-O\-M\-E\-T\-H\-I\-N\-G. For example, {\ttfamily combinations} is a typedef of {\ttfamily basic\-\_\-combinations$<$int$>$}, and {\ttfamily partitions} is a typedef of {\ttfamily basic\-\_\-partitions$<$int$>$}. 